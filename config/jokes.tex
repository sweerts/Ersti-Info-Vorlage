\fancyfoot[C]{%
	\small
	\begin{tikzpicture}[remember picture,overlay]
		\druckrand
	\end{tikzpicture}
	\color{fscolor}
	\flushbottom
	\ifcase\value{page}%
	% empty for page 0
	\or Was ist 2$\pi$ in der Früh? Morgentau.
	\or Was fährt auf Schienen und kehrt die Exponentialfunktion um? -- Eine Dampf-log.
	\or Kommt eine Mathematikstudentin in ein Fotogeschäft: \enquote{Guten Tag! Ich möchte diesen Film entwickeln lassen.} Verkäufer: \enquote{9$\times$13?} -- Studentin: \enquote{117. Wieso?} Kommt ein Mathematik-Professor in ein Fotogeschäft: \enquote{Guten Tag! Ich möchte diesen Film entwickeln lassen.} Verkäufer: \enquote{9$\times$13?} -- Professor: \enquote{Ja, das ist lösbar. Wieso?}
	\or Geht ein Neutron in die Disco, sagt der Türsteher: \enquote{Sorry, heute nur für geladene Gäste!}
	\or Es gibt drei Sorten von Menschen: Solche, die bis drei zählen können, und solche, die nicht bis drei zählen können.
	\or Wieso sind Hausdorff-Räume unsolidarisch? -- Jeder ist sich selbst der Nächste.
	\or Zwei Folgenglieder haben ein Date und nähern sich mit zunehmender Zeit immer mehr einander an. Da ergreift das eine die Initiative und fragt: \enquote{Voulez-vous Cauchy avec moi?}
	\or Der Vorlesung zum Satz von Bolzano-Weierstraß konnte ich nur zum Teil folgen.
	\or Was ist organisierte algebraische Kriminalität? -- Ein Verbrecherring.
	\or Die Mengenoperation $\setminus$ ist so charmant, sie macht mir immer so liebe Komplemente.
	\or Wie nennt man ein ostdeutsches Computermodell? -- Thüring-Maschine!
	\or Wer hat die Differentialgeometrie erfunden? -- Manni G. Faltigkeit.
	% Seite 10
	\or Fragt ein Mathematiker den anderen: \enquote{Ey, wie hoch ist diese Schranke?} Der andere klettert rauf, misst, kommt runter und sagt: \enquote{4,32 Meter.} Sagt der Erste: \enquote{Bist du doof! Warum hast du nicht gewartet, bis die Schranke runter kommt?} Sagt der andere: \enquote{Nee, du bist doof, ich wollte ja wissen wie hoch sie ist, nicht wie breit!}
	\or Treffen sich zwei Funktionen in der Unendlichkeit. Sagt die eine: \enquote{Ich differenzier dich gleich!} Sagt darauf die andere: \enquote{Ätsch, ich bin die e-Funktion.}
	\or Wie ich 2, 3, 5, 7, 11,… finde? Prima!
	\or Wie nennt man eine leichte Kurvendiskussion? -- Banalysis.
	\or Was berechnen Topologen an Weihnachten? Ho-ho-homotopiegruppe.
	\or Ein Gedicht. $\mathbb{Z}$ ist fromm. $\mathbb{Q}$ ist es nicht. Denn $\mathbb{Q}$ ist dicht.
	\or ($\mathbb{Z}$,$+$) und ($\mathbb{Q}\setminus\{0\}$,$\cdot$) gehen ins Kino. $\mathtt{Sym}(3)$ hat eigentlich keine Lust, geht aber trotzdem mit. Gruppenzwang.
	\or ++ Hinrichtung: Mathematikerin stirbt bei Beweis von Äquivalenz ++
	\or Wo sitzt ein stalkender Graphentheoretiker? Auf einem Spannbaum.
	% Seite 20	
	\or Für den Beweis dieses Satzes brauchen wir zwei Hilfssätze. Welch Dilemma.
	\or Vier singende Informatiker bilden einen … Quad-Chor.
	\or Was sind 10 Physiker in Salzsäure? Ein gelöstes Problem!
	\or Was sagt ein Mathematiker zu seiner Frau, nachdem er sie im Bett so richtig scharf gemacht hat? \enquote{Der Rest ist trivial, den kannst du dir als Übungsaufgabe selbst herleiten.}
	\or Was ist der Unterschied zwischen einem Mathematik- und einem Informatikstudenten? Der Mathestudent wollte Mathe studieren.
	\or\ 
	\or\ 
	\or Was ist der Lieblingsfilm einer jeden Mathematikerin? Das Schweigen der Lemma.
	\or Behauptung: Eine Katze hat neun Schwänze. Beweis: Keine Katze hat acht Schwänze. Eine Katze hat einen Schwanz mehr als keine Katze. Deshalb hat eine Katze neun Schwänze.
	\or Ein theoretischer Physiker im Zug fragt die Schaffnerin: \enquote{Entschuldigung, hält an diesem Zug auch Genf?}
	% Seite 30
	\or Eine Mathematikerin ist kurz davor, das erste Mal mit einem Flugzeug zu fliegen. Sie hat wahnsinnig viel Angst -- es könnte ja eine Bombe an Bord sein. Dann hat die Mathematikerin eine Idee: Sie nimmt selbst eine Bombe mit. Die Wahrscheinlichkeit, dass zwei Bomben in einem Flugzeug sind, ist wesentlich geringer, als dass eine Bombe im Flugzeug ist.
	\or Während der Vorlesung soll ein Mathematikprofessor einmal auf die schwierige Aufgabe 7$\times$9 gestoßen sein. Er bittet die Studierenden um Hilfe. Einer ruft: \enquote{62}, eine andere \enquote{65}. Darauf der Professor: \enquote{Aber das ist doch unmöglich! 7$\times$9 kann doch nur 62 ODER 65 sein!}
	\or Wie fängt ein Mathematiker in der Wüste einen Löwen? Er baut einen Käfig, setzt sich rein und definiert: 'Hier ist außen!'
	\or Kommt ein Vektor zur Drogenberatung: \enquote{Hilfe, ich bin linear abhängig!}
	\or Eine Ingenieurin denkt, dass Gleichungen eine Annäherung an die Realität sind. Ein Physiker denkt, dass die Realität eine Annäherung an die Gleichungen ist. Einem Mathematiker ist es egal.
	\or Die Ehe der Professorin soll sehr unglücklich sein, habe ich gehört! -- \enquote{Wundert mich nicht. Sie ist Mathematikerin und ihr Mann unberechenbar.}
	\or Mitten im mathematischen Vortrag erhebt einer der Anwesenden die Hand und sagt: \enquote{Ich habe zu dem, was Sie hier erzählen, ein Gegenbeispiel!} Darauf die Vortragende: \enquote{Egal, ich habe zwei Beweise!}
	\or Was ist die Lieblingsbeschäftigung von Bits? Bus fahren!
	\or Warum werden bei BMW neuerdings keine Mathematikerinnen mehr beschäftigt? Die haben allgemein ein Auto mit $n$ Rädern konstruiert und erst danach den Spezialfall $n=4$ betrachtet.
	% Seite 40
	\or Es gibt 10 Sorten von Menschen. Die einen verstehen Binärcode, die anderen nicht.
	\or\ 
	\or Ein Statistiker kann seinen Kopf in den Backofen und seine Füße in Eiswasser stecken, und er wird sagen: \enquote{Im Durchschnitt geht es mir gut.}
	\or Die Aufgaben in der Prüfung werden die gleichen wie im Kurs sein. Es werden nur die Zahlen verändert. Aber keine Sorge: Pi bleibt 3.141592…
	\or Wie besteigt eine Mathematikerin den Mount Everest? Sie integriert mit Hilfe einer Treppenfunktion über den Berg und steigt sie dann hinauf.
	\or Studentin: \enquote{Herr Professor, können Sie uns zu diesem Beweis auch ein Beispiel vorrechnen?} Professor: \enquote{Mit diesem Beweis habe ich Ihnen bereits alle Beispiele vorgerechnet.}
	\or Treffen sich zwei Geraden. Sagt die eine: \enquote{Beim nächsten Mal gibst du einen aus.}
	\or Abiturprüfung. Schulleiter zum Abiturienten: \enquote{Kennen wir uns nicht?} Abiturient: \enquote{Ja, vom Mathe-Abi im letzten Jahr.} Schulleiter: \enquote{Ach so, ja. Aber heute wird's schon klappen. Wie lautete denn damals die erste Frage, die ich Ihnen gestellt habe?} Abiturient: \enquote{Kennen wir uns nicht…}	
	\or Wie bringen Mathematiker*innen ihre Gegner um, ohne eine Mordwaffe zu hinterlassen? Sie legen ihnen einen Kreis um den Hals und lassen den Radius gegen null gehen.
	\or Warum konnten Seeräuber keine runden Kanonenkugeln herstellen? Na, weil sie Pi raten!
	% Seite 50
	\or Werner Heisenberg wird auf der Autobahn von der Polizei angehalten. Die Beamtin verlangt nach Führer- und Fahrzeugschein, schaut sich diese an und fragt: \enquote{Herr Heisenberg, wissen Sie, wie schnell Sie gefahren sind?} \enquote{Nein}, antwortet Heisenberg, \enquote{aber ich weiß, wo ich jetzt bin!}
	\or Wie viele Quantenmechaniker braucht man, um eine Glühbirne zu wechseln? Man braucht einen Quantenmechaniker, um die Glühbirne wahrscheinlich zu wechseln.
	\or Der Computer löst Probleme, die man ohne ihn nicht hätte.
	\or Treffen sich zwei Matrizen. Sagt die eine: \enquote{Komm wir gehen in den Wald und machen A hoch minus 1.} Sagt die andere: \enquote{Mensch, bist Du invers!}
	\or Eine Mathematikerin will ihren neuesten Beweis als Bild aufhängen. Sie nimmt Nagel und Hammer und hält den Nagel mit dem Kopf zur Wand. Gerade als sie zuschlagen will, schaut sie noch mal genau hin -- und stutzt. Nach fünf Minuten konzentrierten Hinschauens und Überlegens hat sie's: \enquote{Das ist ein Nagel für die gegenüberliegende Wand!}
	\or Wusstest du, dass fast alle Menschen mehr Beine haben als der Durchschnitt?
	\or Jede natürliche Zahl ist interessant, denn angenommen es gäbe uninteressante natürliche Zahlen. Dann gäbe es auch eine kleinste uninteressante Zahl, und das machte diese Zahl furchtbar interessant!
	\or Gespräch zweier Informatikerinnen: \enquote{Wie ist denn das Wetter bei euch?} -- \enquote{Caps Lock.} -- \enquote{Hä?} -- \enquote{Na ja, Shift ohne Ende!}
	\or Was sagt ein arbeitsloser Mathematiker zu einem Mathematiker, der gerade Arbeit gefunden hat? \enquote{Einmal Pommes mit Mayo bitte!}
	% Seite 60	
	\or Why didn't Newton discover group theory? Because he wasn't Abel.
	%\or Wie oft kann man 7 von 83 abziehen, und was bleibt am Ende übrig? Man kann so oft wie man will 7 von 83 abziehen, und es bleibt jedes Mal 76 über.
	\or What is a bird's favourite type of maths? Owl-gebra.
	\or If debugging is the process of removing bugs, then programming must be the process of putting them in.
	
	\or Party im Raum der stetigen Funktionen. Sinus und Cosinus tanzen wild auf und ab, die Polynome bilden einen Ring. Alle anwesenden Funktionen amüsieren sich prächtig, nur $e^x$ steht alleine in der Ecke. Da kommt $x^2$ vorbei und meint: \enquote{Mensch, jetzt integrier dich doch einfach mal.} $e^x$ darauf traurig: \enquote{Hab ich ja schon, aber das hat auch nix geändert.}
	\or Der Computer rechnet mit allem -- nur nicht mit seinem Besitzer.
	\or Was ist denn mit Deiner süßen kleinen Freundin, der Mathematikerin? -- \enquote{Die habe ich verlassen. Ich rufe sie an -- da erzählt sie, dass sie im Bett liegt und sich mit 3 Unbekannten rumplagt…}
	\or Die meistgestellten Fragen: Ingenieur: Wie geht das? Ökonom: Wie teuer wird das? Mathematiker: Wie kann man das verbessern? Physiker: Möchten Sie dazu Ketchup?
	\or Ein Mathelehrer steht vor der Klasse und erklärt: \enquote{Es gibt keine größere und keine kleinere Hälfte. Aber warum erzähl ich euch das überhaupt, die größere Hälfte von euch versteht das ja doch nicht.}
	\or Wie viel ist dreimal sieben? GANZ feiner Sand! Und was ist viermal sechs? Anstrengend…
	% Seite 70	
	\or Ein Mathematikstudent kommt mit einem nagelneuen Fahrrad in die Uni gefahren. Sofort fragen ihn seine Kommilitonen, woher er es hat. \enquote{Ich fahre so durch den Park, als plötzlich ein Mädchen von ihrem Fahrrad springt, sich auszieht und meint, ich könne alles von ihr haben.} Darauf seine Mathe-Kommilitonen: \enquote{Echt gute Wahl, die Klamotten hätten Dir sowieso nicht gepasst!}
	\or Die Mathelehrerin sagt: \enquote{Die Klasse ist so schlecht in Mathe, dass sicher 90\% dieses Jahr durchfallen werden.} Ein Schüler im Hintergrund: \enquote{Aber so viele sind wir doch gar nicht!}
	\or Warum verwechseln Informatiker Weihnachten immer mit Halloween? Weil \texttt{OCT 31} gleich \texttt{DEC 25} ist.
	\or Welches Tier kann addieren? Na, ein Oktoplus!
	\or Ingenieurin zum Mathematiker: \enquote{Ich finde Ihre Arbeit ziemlich monoton.} Mathematiker: \enquote{Mag sein, dafür ist sie aber stetig und nicht beschränkt.}
	\or Ein Mathematiker ist ein Gerät, welches Kaffee in Behauptungen umwandelt.
	\or Kommt ein Nullvektor zum Psychiater: \enquote{Herr Doktor, ich bin immer so orientierungslos!}
	\or Eine Soziologin, ein Physiker und eine Mathematikerin fahren im Zug. Sie schauen aus dem Fenster und sehen ein schwarzes Schaf. Soziologin: \enquote{Hier gibt es schwarze Schafe.} Physiker: \enquote{Falsch. Hier gibt es mindestens ein schwarzes Schaf.} Mathematikerin: \enquote{Immer noch falsch. Hier gibt es mindestens ein Schaf, das auf mindestens einer Seite schwarz ist.}
	\or Warum schauen Mathematiker so gerne Seifenopern im Fernsehen? Es gibt jeden Tag eine neue Folge.
	% Seite 80
	\or Eine Statistikerin wird gefragt, wo sie begraben werden will. Seine Antwort: \enquote{In Jerusalem, da ist die Auferstehungswahrscheinlichkeit am größten.}
	\or Treffen sich zwei Pointer auf dem Stack. Sagt der eine zum anderen: \enquote{Hör auf, auf mich zu zeigen!}
	\or Mathematiker sterben nicht, sie verlieren nur einige ihrer Funktionen.
	\or Eine Physikerin untersucht die Fallgeschwindigkeit eines Thermometers. Sie lässt ein Thermometer und ein Wachslicht gleichzeitig fallen und bemerkt, dass beide gleichzeitig unten ankommen. Schlussfolgerung: Das Thermometer fällt mit der Geschwindigkeit von Licht.
	\or Warum sind Birnen auch Homomorphismen? Sie haben Kerne.
	\or Eine Ingenieurin, ein theoretischer und ein Experimentalphysiker wachen nachts auf und merken, dass ihre Häuser brennen. Was tun sie? Die Ingenieurin rennt zum Feuerlöscher, löscht damit den Brand und legt sich wieder schlafen. Der theoretische Physiker setzt sich an den Schreibtisch, rechnet, nimmt dann ein Glas Wasser und schüttet es so auf das Feuer, dass es erlischt. Der Experimentalphysiker verbrennt auf der Suche nach einem Thermometer.
	\or Wenn du einen Mathematiker wählen lässt zwischen einem Brötchen und ewiger Seligkeit, was nimmt er? Natürlich das Brötchen: Nichts ist besser als ewige Seligkeit und ein belegtes Brötchen ist besser als nichts.
	\or Woran erkennt man, dass eine Zahnärztin früher einmal Mathematikerin war? Das Einzige, was sie tut, ist Wurzelziehen.
	\or Was ist Pi? Mathematiker: \enquote{\pi ist die Zahl, die das Verhältnis vom Umfang eines Kreises und seinem Durchmesser angibt.} Physikerin: \enquote{\pi ist 3,1415927 plus/minus 0,00000005.} Ingenieur: \enquote{\pi ist ungefähr 3.}
	% Seite 90
	\or Was ist gelb, krumm, normiert und vollständig? Ein Bananachraum!
	\or Auf der Heizung liegt ein Ziegelstein. Prüferin: \enquote{Warum ist der Stein auf der Heizung abgewandten Seite wärmer?} Prüfling: \enquote{Äh, vielleicht wegen Wärmeleitung und so?} Prüferin: \enquote{Nein, weil ich ihn gerade umgedreht habe.}
	\or Was haben eine Mathematikerin und ein Physiker gemeinsam? Beide sind dumm -- mit Ausnahme der Mathematikerin.
	\or Sitzt ein Mathematiker in der Kneipe und saugt am Rand seines Glases. Da kommt ein zweiter Mathematiker vorbei und fragt, warum er denn nicht wie alle anderen trinke. Darauf der erste: \enquote{Nach dem Satz von Gauß muss das auch so klappen.}
	\or Prüfer: \enquote{Malen Sie doch mal eine Skizze vom Sinus.} (Prüfling malt.) Prüfer: \enquote{Sieht doch schon ganz gut aus.} Student: \enquote{Nein, das sollte die x-Achse sein, ich bin so aufgeregt.}
	\or Frau Meier will ihrer Nachbarin zeigen, wie toll ihr Sohn Fritz schon rechnen kann: \enquote{Fritz, was ist drei mal vier?} -- \enquote{Zehn!} -- \enquote{Sehen Sie, nur um eins verrechnet!}
	\else
	Too many pages... ;)
	\fi
}